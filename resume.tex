%% RESUME - Philip Linden
\documentclass[10pt,final,sans]{resume}

% --------
% preamble
% ========

% --------

\begin{document}
\setlength\headheight{28pt} % make header tall enough
\name{PHILIP J. LINDEN}
\lcontact{
  \begin{tabular}{@{}ll@{}}
    \faGlobe&\href{https://runphilrun.github.io/}{runphilrun.github.io} \\
    \faLinkedin&\href{https://www.linkedin.com/in/philiplinden/}{philiplinden} \\
    \faAngellist&\href{https://angel.co/philip-linden}{philip-linden} \\
    \faGithub&\href{https://github.com/runphilrun/}{runphilrun}
  \end{tabular}
}
\rcontact{
  \begin{tabular}{@{}r@{}}
    pjl7651@rit.edu \\
    440 N. Winchester Blvd. Apt 128 \\
    Santa Clara, CA 95050 \\
    (585) 690-7067
  \end{tabular}
}

\section{Professional Summary}
I am a recent graduate who is passionate about the design and analysis of
aviation and space systems, including but not limited to satellites, human
spaceflight, spacecraft and aircraft structures, propulsion, mechanisms,
imaging, and controls. I am relentlessly curious, a strong visionary, and
optomistic about the future of technology and humankind.

\section{Degree}
\headerwithlabel{Rochester Institute of Technology}{Rochester, NY}{Aug 2012 --
  May 2017}
\begin{tabular}{ll}
Bachelor of Science in Mechanical Engineering -- Aerospace Option & {\bf GPA: } 3.5 \\
Master of Engineering in Mechanical Engineering (Dual Degree) & {\bf GPA: } 3.3 \\
\end{tabular}

\section{Engineering Experience}
\headerwithlabel{Lockheed Martin Space}{Sunnyvale, CA}{June 2017--Present}
Electro-Optical Engineer, Optical Payload Center of Excellence
\begin{itemize}
  \item Planned and conducted experiments and analysis to characterize focal plane arrays.
  \item Led a software team through critical development milestones for a Matlab engineering tool. 
  \item Led a team to win a company-wide machine learning hackathon with Python and sci-kit learn.
\end{itemize} 
\headerwithlabel{SpaceX}{Hawthorne, CA}{June--August 2016}
Vehicle Engineering Intern, Capsule Structures

\headerwithlabel{RIT Center for Detectors}{Rochester, NY}{March--May 2016}
Lab Assistant, Mechanical Engineer

\headerwithlabel{SpaceX}{Hawthorne, CA}{January--July 2015}
Vehicle Engineering Intern, Capsule Reusability

\headerwithlabel{GE Aviation}{Cincinnati, OH}{January--May 2014}
Engineering Co-op, Ultrasonic Non-Destructive Test Lab

% \section{Additional Experience}
% \headerwithlabel{RIT Undergraduate Admissions}{Rochester, NY}{Fall 2013--May 2017}

\section{Projects}
\headerwithlabel{Cosmic Dawn Intensity Mapper System-Level Design}{\small\href{https://github.com/runphilrun/CDIM-design/blob/master/cdim_design.pdf}{github.com/runphilrun/CDIM-design}}{}
Contributed to a proposal for a Probe Class (\textasciitilde\$850M) NASA mission for a 1.5
meter space telescope intended to observe near-infrared light from the early
universe.
\begin{itemize}
  \item Compiled financial, mass, and power budgets for the optics, instruments,
    cryocooler \& spacecraft.
  \item Defined system-level design, generated representative CAD models and
    figures for the entire spacecraft.
\end{itemize}

\headerwithlabel{1 kW Arcjet Thruster}{\small\href{https://github.com/RIT-Space-Exploration/msd-P17101/blob/master/p17101.pdf}{github.com/RIT-Space-Exploration/msd-P17101}}{}
Developed the concept, system-level design, and nozzle design for a small scale
arcjet thruster demonstration. Worked in a multidisciplinary team of mechanical and electrical engineers.
\begin{itemize}
  \item Responsible for communication between the team and the customer (RIT Space Exploration).
  \item Designed and performed CFD analysis on the thruster nozzle.
\end{itemize}


\end{document}