%% RESUME - Philip Linden
\documentclass[10pt,final,sans]{resume}

\begin{document}
\setlength\headheight{28pt} % make header tall enough
\name{PHILIP J. LINDEN}
\lcontact{
  \begin{tabular}{@{}ll@{}}
    \faLinkedin & \href{https://www.linkedin.com/in/philiplinden/}{philiplinden} \\
    \faAngellist & \href{https://angel.co/philip-linden}{philip-linden} \\
    \faGithub & \href{https://github.com/runphilrun/}{runphilrun} \\
    \faGlobe & \href{https://runphilrun.github.io/}{runphilrun.github.io} 
  \end{tabular}
}
\rcontact{
  \begin{tabular}{@{}r@{}}
    pjl7651@rit.edu \\
    440 N. Winchester Blvd. Apt 128 \\
    Santa Clara, CA 95050 \\
    (585) 690-7067
  \end{tabular}
}

\section{Professional Summary}
I am an engineer who is passionate about the design and analysis of
aviation and space systems, including but not limited to satellites, human
spaceflight, spacecraft and aircraft structures, propulsion, mechanisms, remote sensing, 
imaging, and controls. I am relentlessly curious, a strong visionary, and
optomistic about the future of technology and humankind.

\section{Degrees}
\headerwithlabel{Rochester Institute of Technology}{Rochester, NY}{May 2017}
Bachelor of Science in Mechanical Engineering -- Aerospace Option \\
Master of Engineering in Mechanical Engineering (Dual Degree) \\
{\bf Graduate Paper:} {\href{https://github.com/runphilrun/CDIM-design/blob/master/cdim_design.pdf}{Cosmic Dawn Intensity Mapper (CDIM)} \\
{\bf Undergraduate Capstone:} \href{https://github.com/RIT-Space-Exploration/msd-P17101/blob/master/p17101.pdf}{1 kW Arcjet Thruster}

\section{Technical Skills}
Systems Engineering, Mechanical Engineering, Aerospace Engineering, Electro-Optical Engineering, Image Processing, MATLAB, Python, OpenCV, Git, {\textrm \LaTeX}, Solidworks, NX8.5, MS Project, MS Visio, MS Office Superuser, Linux, Controls

\section{Engineering Experience}
\headerwithlabel{Lockheed Martin Space}{Sunnyvale, CA}{June 2017 -- Present}
Electro-Optical Engineer, Optical Payload Center of Excellence
\begin{itemize}
  \item Characterize focal plane arrays and imaging systems in optical labs.
  \item Systems engineering and Electro-Optical engineering on IRAD projects to support major business pursuits.
  \item Led a software team through critical development milestones for Matlab engineering tools. 
  \item {\bf Projects:} Machine Learning Hackathon, Optical Payload Training Course Project
\end{itemize} 
\headerwithlabel{SpaceX}{Hawthorne, CA}{June -- August 2016}
Vehicle Engineering Intern, Capsule Structures
\begin{itemize}
  \item Modeled and drafted designs for critical structures for the Crew Dragon vehicle.
  \item {\bf Projects:} Crew Dragon Weldment Doubler
\end{itemize}

Vehicle Engineering Intern, Capsule Reusability \hfill January -- July 2015
\begin{itemize}
  \item Project development, including hands-on prototyping and designing, conducting and \\
  presenting experiments to explore changes to Dragon Cargo space capsules.
  \item Approached several projects simultaneously which demanded intensive problem-solving, \\
  interpersonal, and time management skills.
  \item {\bf Projects:} Dragon Capsule Water Sealing, Dragon Capsule Parachute Packing Tool Rework
\end{itemize}

\headerwithlabel{RIT Center for Detectors}{Rochester, NY}{March -- May 2016}
Lab Assistant, Mechanical Engineer
\begin{itemize}
  \item Created system-level designs and modeled mechanical components for astronomy research \\ 
  experiments including a cryogenic sounding rocket payload, a ground-based observatory telescope, \\
  and small spacecraft.
  \item Led a team of undergraduate students and served as systems engineer for integration of a \\
  NASA sounding rocket research payload.
  \item {\bf Projects:} Cryogenic Star Tracking Attitude Regulation System (CSTARS)
\end{itemize}

\headerwithlabel{GE Aviation}{Cincinnati, OH}{January -- May 2014}
Engineering Co-op, Ultrasonic Non-Destructive Test Lab
\begin{itemize}
  \item Operated ultrasonic transducers and 3-axis scanners.
  \item Analyzed scan imagery for component defects in test samples and flight hardware, including \\
  composite delaminations and weld voids.
  \item Developed and optimized test procedures for components with irregular geometry.
  \item {\bf Projects:} GEnx Flowpath Spacer Inspection Optimization
\end{itemize}

\headerwithlabel{RIT Space Exploration (RITSPEX)}{Rochester, NY}{Fall 2014 -- Present}
Alumni Member
\begin{itemize}
  \item Mentor undergraduate students working on space exploration projects.
  \item Principal Investigator and Project Lead for computer vision and remote sensing payloads.
  \item {\bf Projects:} SPEX Project Definition Document Template, {\it Where U At Plants?}~(WUAP) HAB Payload, SPEXcast Podcast
\end{itemize}

% \section{Additional Experience}
% \headerwithlabel{RIT Undergraduate Admissions}{Rochester, NY}{Fall 2013--May 2017}

\break
\section{Detailed Project Descriptions}

\headerwithlabel{Cosmic Dawn Intensity Mapper (CDIM)}{\href{https://github.com/runphilrun/CDIM-design/blob/master/cdim_design.pdf}{github.com/runphilrun/CDIM-design}}{\bf Graduate Paper}
Contributed to a proposal for a Probe Class (\textasciitilde\$850M) NASA mission for a 1.5 meter space telescope intended to observe near-infrared light from the early universe. Compiled financial, mass, and power budgets for the optics, instruments, cryocooler \& spacecraft. Defined system-level design, generated representative CAD models and figures of the spacecraft.

\headerwithlabel{1 kW Arcjet Thruster}{\href{https://github.com/RIT-Space-Exploration/msd-P17101/blob/master/p17101.pdf}{github.com/RIT-Space-Exploration/msd-P17101}}{\bf Undergraduate Capstone}
Developed the concept, system-level design, and nozzle design for a small scale arcjet thruster demonstration. Worked in a multidisciplinary team of mechanical and electrical engineers. Responsible for communication between the team and the customer (RIT Space Exploration). Designed and performed CFD analysis on the thruster nozzle.

\headerwithlabel{Where U At Plants?~(WUAP) High Altitude Balloon Payload}{\href{https://github.com/RIT-Space-Exploration/hab-cv/}{github.com/RIT-Space-Exploration/hab-cv}}{}
Where U At Plants? (WUAP) is a high-altitude balloon payload using on-board image processing with a Raspberry Pi 3, Python 3 and OpenCV 3.3 to mask RGB images of the Earth and attempts to mask areas of vegetation using colorspace transformations. WUAP few as a payload on RIT Space Exploration's HAB4 high altitude balloon mission on April 22, 2018. A \href{https://github.com/RIT-Space-Exploration/hab-cv/blob/master/reports/Project%20Definition%20Document/hab-cv.pdf}{Project Definition Document} and \href{https://github.com/RIT-Space-Exploration/hab-cv/blob/master/reports/HAB4%20Post%20Flight%20Report/report_wuap_postflight-hab4.md}{post-flight report} document the design intent and discuss the results.

\headerwithlabel{Cryogenic Star Tracking Attitude Regulation System (CSTARS)}{RIT Center for Detectors}{}
Designed the mechanical model of CSTARS, an experiment endorsed by the New York Space Grant and funded with \$100,000 by NASA's Undergraduate Student Instrument Program. I designed CAD models for the cryogenic thermal regulation system, telescope, and mechanical supports in Solidworks 2015. I was the systems engineer for payload integration with a Black Brant IX at NASA Wallops Flight Facility.

\headerwithlabel{SPEX Project Definition Document Template}{\small\href{https://github.com/RIT-Space-Exploration/SPEX-Project-Definition-Documents}{github.com/RIT-Space-Exploration/SPEX-Project-Definition-Documents}}{}
A Project Definition Document (PDD) documents a SPEX project idea and its objectives. This template defines the internal standard of quality for all SPEX PDDs and also serves as a template for other projects to build from. This template was used to define the scope of RIT's payload for the Intercollegiate Rocket Engineering Competition 2018, earning the project a \$1000 grant from Students for the Exploration and Development of Space (SEDS).

\headerwithlabel{Machine Learning Hackathon}{Lockheed Martin Space}{}
Led a team to win a company-wide machine learning hackathon competition. Implemented Expectation Maximization algorithm using K-means in Python3 with sci-kit learn. Presented the project approach and results to a panel of LM Engineering \& Technology upper management. 

\headerwithlabel{Optical Payload Training Course Project}{Lockheed Martin Space}{}
Led a multidisciplinary team in a course project to design an optical payload mission concept and instrument. Worked in a multidisciplinary team with subject matter experts in Sunnyvale, CA and Denver, CO. Coordinated team meetings, managed progress and action items. Developed an atmospheric science mission concept and designed the instrument.

\headerwithlabel{Dragon Capsule Water Sealing}{SpaceX}{}
Designed and tested retrofits to the Dragon Cargo capsule in order to prevent water ingress on splashdown. Investigated water entry paths, conducted experiments to validate designs, and implemented modifications on flight hardware present on Dragon vehicles since the CRS-7 mission.

\headerwithlabel{Dragon Capsule Parachute Packing Tool Rework}{SpaceX}{}
Designed and implemented modifications to the Dragon Cargo parachute packing tool, including working with third party vendors to deliver flight critical components.

\headerwithlabel{Crew Dragon Weldment Doubler Design}{SpaceX}{}
Designed and drafted engineering CAD models and drawings for structural components installed on critical load members of a flight vehicle. Design start to installation took 4 days.

\headerwithlabel{GEnx Flowpath Spacer Inspection Optimization}{GE Aviation}{}
Optimized parameters for detection of internal wrinkles in composite layups with complex geometry during ultrasonic inspection. Conducted destructive microscopy to validate results and presented findings to principal engineers.

\headerwithlabel{SPEXcast Podcast}{\href{https://blog.spexcast.com/}{blog.spexcast.com}}{}
I produce, edit, and co-host a space exploration podcast, which is a weekly discussion podcast the science and technology of space exploation. SPEXcast also features interviews with space scientists and industry members, including Tory Bruno, Chris Hadfield, and NASA Scientists.

\end{document}