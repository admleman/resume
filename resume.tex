%% RESUME - Philip Linden
\documentclass[10pt,final,sans]{resume}

% --------
% preamble
% ========

% --------

\begin{document}
\setlength\headheight{28pt} % make header tall enough
\name{PHILIP J. LINDEN}
\lcontact{
  \begin{tabular}{@{}ll@{}}
    \faLinkedin&\href{https://www.linkedin.com/in/philiplinden/}{philiplinden} \\
    \faAngellist&\href{https://angel.co/philip-linden}{philip-linden} \\
    \faGithub&\href{https://github.com/runphilrun/}{runphilrun} \\
    \faGlobe&\href{https://runphilrun.github.io/}{runphilrun.github.io} 
  \end{tabular}
}
\rcontact{
  \begin{tabular}{@{}r@{}}
    pjl7651@rit.edu \\
    440 N. Winchester Blvd. Apt 128 \\
    Santa Clara, CA 95050 \\
    (585) 690-7067
  \end{tabular}
}

\section{Professional Summary}
I am an engineer who is passionate about the design and analysis of
aviation and space systems, including but not limited to satellites, human
spaceflight, spacecraft and aircraft structures, propulsion, mechanisms, remote sensing, 
imaging, and controls. I am relentlessly curious, a strong visionary, and
optomistic about the future of technology and humankind.

\section{Degrees}
\headerwithlabel{Rochester Institute of Technology}{Rochester, NY}{May 2017}
Bachelor of Science in Mechanical Engineering -- Aerospace Option \\
Master of Engineering in Mechanical Engineering (Dual Degree) 

\section{Engineering Experience}
\headerwithlabel{Lockheed Martin Space}{Sunnyvale, CA}{June 2017 -- Present}
Electro-Optical Engineer, Optical Payload Center of Excellence
\begin{itemize}
  \item Characterize focal plane arrays and imaging systems in optical labs.
  \item Led a software team through critical development milestones for Matlab engineering tools. 
\end{itemize} 
\headerwithlabel{SpaceX}{Hawthorne, CA}{June -- August 2016}
Vehicle Engineering Intern, Capsule Structures
\begin{itemize}
  \item nothing is here
\end{itemize}

Vehicle Engineering Intern, Capsule Reusability \hfill January -- July 2015
\begin{itemize}
  \item Project development, including hands-on prototyping and designing, conducting and \\
  presenting experiments to explore changes to Dragon Cargo space capsules.
  \item Approached several projects simultaneously which demanded intensive problem-solving, \\
  interpersonal, and time management skills.
\end{itemize}

\headerwithlabel{RIT Center for Detectors}{Rochester, NY}{March -- May 2016}
Lab Assistant, Mechanical Engineer
\begin{itemize}
  \item Created system-level designs and modeled mechanical components for astronomy research \\ 
  experiments including a cryogenic sounding rocket payload, a ground-based observatory telescope, \\
  and small spacecraft.
  \item Led a team of undergraduate students and served as systems engineer for integration of a \\
  NASA sounding rocket research payload.
\end{itemize}
\headerwithlabel{GE Aviation}{Cincinnati, OH}{January -- May 2014}
Engineering Co-op, Ultrasonic Non-Destructive Test Lab
\begin{itemize}
  \item Operated ultrasonic transducers and 3-axis scanners.
  \item Analyzed scan imagery for component defects in test samples and flight hardware, including \\
  composite delaminations and weld voids.
  \item Developed and optimized test procedures for components with irregular geometry.
\end{itemize}

% \section{Additional Experience}
% \headerwithlabel{RIT Undergraduate Admissions}{Rochester, NY}{Fall 2013--May 2017}
\section{Organizations}
\headerwithlabel{RIT Space Exploration (RITSPEX)}{Alumni Member}{Fall 2014 -- Present}
Actively participate in RIT SPEX

\section{Skills}
Systems Engineering, Mechanical Engineering, Aerospace Engineering, Image Processing, MATLAB, Python, OpenCV, Git, {\textrm \LaTeX}, Solidworks, MS Project, MS Visio, MS Office Superuser, Linux, Technical Writing, Controls

\break
\section{Projects}
\headerwithlabel{Cosmic Dawn Intensity Mapper (CDIM)}{\href{https://github.com/runphilrun/CDIM-design/blob/master/cdim_design.pdf}{github.com/runphilrun/CDIM-design}}{\bf Graduate Paper}
Contributed to a proposal for a Probe Class (\textasciitilde\$850M) NASA mission for a 1.5 meter space telescope \\
intended to observe near-infrared light from the early
universe.
\begin{itemize}
  \item Compiled financial, mass, and power budgets for the optics, instruments,
    cryocooler \& spacecraft.
  \item Defined system-level design, generated representative CAD models and
    figures for the entire spacecraft.
\end{itemize}

\headerwithlabel{1 kW Arcjet Thruster}{\href{https://github.com/RIT-Space-Exploration/msd-P17101/blob/master/p17101.pdf}{github.com/RIT-Space-Exploration/msd-P17101}}{\bf Undergraduate Capstone}
Developed the concept, system-level design, and nozzle design for a small scale arcjet thruster demonstration. 
\begin{itemize}
  \item Worked in a multidisciplinary team of mechanical and electrical engineers.
  \item Responsible for communication between the team and the customer (RIT Space Exploration).
  \item Designed and performed CFD analysis on the thruster nozzle.
\end{itemize}

\headerwithlabel{Where U At Plants?~(WUAP) High Altitude Balloon Payload}{\href{https://github.com/RIT-Space-Exploration/hab-cv/}{github.com/RIT-Space-Exploration/hab-cv}}{}
nothing is here

\headerwithlabel{Machine Learning Hackathon}{Lockheed Martin Space}{}
Led a team to win a company-wide machine learning hackathon competition.
\begin{itemize}
  \item Implemented Expectation Maximization algorithm using K-means in Python3 with sci-kit learn.
  \item Presented the project approach and results to a panel of LM Engineering \& Technology upper management.
\end{itemize}

\headerwithlabel{Optical Payload Training Course Project}{Lockheed Martin Space}{}
Led a multidisciplinary team in a course project to design an optical payload mission concept and instrument.
\begin{itemize}
  \item Worked in a multidisciplinary team with subject matter experts in Sunnyvale, CA and Denver, CO. 
  \item Coordinated team meetings, managed progress and action items.
  \item Developed an atmospheric science mission concept and designed the instrument.
\end{itemize}

\headerwithlabel{Cryogenic Star Tracking Attitude Regulation System (CSTARS)}{RIT Center for Detectors}{}
Designed the mechanical model of CSTARS, an experiment endorsed by the New York Space Grant and funded with \$100,000 by NASA's Undergraduate Student Instrument Program. I designed CAD models for the cryogenic thermal regulation system, telescope, and mechanical supports in Solidworks 2015. I was the systems engineer for payload integration with a Black Brant IX at NASA Wallops Flight Facility.

\headerwithlabel{Dragon Capsule Water Sealing}{SpaceX}{}
nothing is here

\headerwithlabel{Dragon Capsule Parachute Packing Tool Rework}{SpaceX}{}
nothing is here

\headerwithlabel{Crew Dragon Weldment Doubler Design}{SpaceX}{}
nothing is here

\headerwithlabel{GEnx Flowpath Spacer Inspection Optimization}{GE Aviation}{}
nothing is here

\headerwithlabel{SPEX Project Definition Document Guide}{\href{https://github.com/RIT-Space-Exploration/SPEX-Project-Definition-Documents}{github.com/RIT-Space-Exploration/SPEX-Project-Definition-Documents}}{}
nothing is here

\headerwithlabel{SPEXcast Podcast}{\href{https://blog.spexcast.com/}{blog.spexcast.com}}{}
nothing is here

\end{document}